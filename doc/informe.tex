\documentclass[12pt]{article}
\pagestyle{plain}
\usepackage{graphicx}
\usepackage{float}
\usepackage[spanish]{babel}
\usepackage [latin1]{inputenc}
\usepackage{verbatim}

\begin{document}

\tableofcontents
\clearpage

\section{Hiptesis y Aclaraciones}

Si al comando Mover se le pasa como destino un directorio, deber mover a ste el archivo de origen con su mismo nombre.

En los directorios utilizados para archivos de datos (\verb|$DATADIR| y sus subdirectorios), no deben existir archivos ordinarios con el nombre "dup", ya que no permitira la creacin de secuencias de duplicados (para lo cual \verb|dup| deber ser un directorio). En rigor de verdad esta situacin podra manejarse, ya sea eliminando los archivos ordinarios de nombre "dup" para crear directorios en su lugar, o descartando los archivos duplicados. Consideramos que no es correcto no almacenar los duplicados, y adems no es necesario tener archivos de nombre "dup", debido a que los archivos de entrada tienen nombres con estructuras definidas que no permiten a "dup" ser un archivo de entrada vlido.

La correcta inicializacin del entorno queda indicada mediante una variable definida a tal propsito (\verb|$ENTORNO_INICIALIZADO|). Basta con chequear desde un comando la existencia de esta variable para asumir un entorno de trabajo vlido.

La ruta relativa de \verb|practico.conf| respecto del comando iniciar debe ser: \verb|../conf/practico.conf|. De esta forma se asegura que iniciar pueda encontrarlo y a partir de l recibir cualquier otra informacin sobre rutas que necesite para continuar su correcta ejecucin.

DETALLAR HIPTESIS ASUMIDAS SOBRE LA ESTRUCTURA DE DIRECTORIOS PARA PERMITIR A INSTALAR ENCONTRAR \verb|practico.conf|.

\section{Problemas relevantes}

DETALLAR PROBLEMAS ENCONTRADOS DURANTE EL DESARROLLO Y LAS PRUEBAS.



\section{Archivo README: Instructivo de instalacin}
Cuando est terminado el README insertar en el \verb|.tex| segn su ruta haciendo:
\begin{verbatim}
\verbatiminput{ruta/README}
\end{verbatim}

\section{Comandos Desarrollados}
\subsection{Comando Mover}
\begin{description}
	\item [Tipo de comando:] Solicitado
	
	\item [Archivos de entrada:] Archivo especificado por el parmetro 1
	
	\item [Archivos de salida:] Archivo o directorio especificado por el parmetro 2; archivo de log del comando que lo invoca: \verb|$LOGDIR/comando.log|
	
	\item [Parmetros:]
	\begin{enumerate}
		\item Archivo de origen a mover
		\item Archivo o directorio de destino
		\item (Opc.) Comando que invoca a mover, permite guardar informacin en su log
	\end{enumerate}
	
	\item [Ejemplos de invocacin:]
	Sea \verb|a1| la ruta a un archivo existente, \verb|d1| un directorio existente, invocando como:
	\begin{verbatim}$ mover a1 d1\end{verbatim}
	el comando mueve el archivo a1 al directorio \verb|d2|, informando que la operacin es exitosa. Por mover se entiende que cambia el directorio padre del archivo origen.\\
	Sea \verb|a1| un archivo existente, \verb|d2| una ruta invlida, invocando como:
	\begin{verbatim}$ mover a1 d2\end{verbatim}
	el comando \verb|mv|, invocado por mover, informa que el directorio de destino no existe.\\
	Sea \verb|a1| y \verb|ruta/a2| archivos existentes, donde \verb|ruta/dup| no existe, invocando como:
	\begin{verbatim}$ mover a1 ruta/a2\end{verbatim}
	el comando crea en ruta el subdirectorio \verb|dup|, y mueve all el archivo \verb|a1| con nombre \verb|a2.1|. Este representa el primer duplicado de \verb|a2|. Si se vuelve a invocar como:
	\begin{verbatim}$ mover ax ruta/a2\end{verbatim}
	el comando mueve \verb|ax| a \verb|ruta/dup/a2.2|. As va formando una secuencia de archivos duplicados para no permitir la sobreescritura.\\
	Sea \verb|a1| un archivo existente, \verb|ruta/a1| otro archivo existente, invocando como:
	\begin{verbatim}$ mover a1 ruta\end{verbatim}
	el comando intenta mover \verb|a1| a \verb|ruta/a1|; al existir este archivo, se aade un duplicado a la secuencia de \verb|a1| (o se la inicia si no existen duplicados previos).\\
	Sea \verb|a3| una ruta invlida, \verb|a4| una ruta vlida o invlida, invocando como:
	\begin{verbatim}$ mover a3 a4 com1\end{verbatim}
	el comando informa que el archivo de origen es inexistente. Adems, por haber includo el tercer parmetro, el comando guarda en \verb|$LOGDIR/com1.log|, a travs del comando \verb|glog| todos los mensajes de informacin y/o error que, al igual que en las dems ejecuciones, tambin enva por la salida estndar.
	
	\item [Cdigo fuente:]
\end{description}
{\footnotesize
\verbatiminput{../bin/mover}
}

\subsection{Comando Iniciar}
\begin{description}
	\item [Tipo de comando:] Solicitado
	
	\item [Archivos de entrada:] El archivo de configuracin \verb|practico.conf|
	
	\item [Archivos de salida:] El archivo de log \verb|$LOGDIR/iniciar.log|
	
	\item [Ejemplo de invocacin:] El comando iniciar debe invocarse como
	\begin{verbatim}$ . iniciar\end{verbatim}
	El comando chequear la presencia en el archivo de configuracin \verb|practico.conf| de las variables indispensables para el correcto funcionamiento de todos los comandos. Solicitar parmetros necesarios para el comando detectar: cantidad mxima de ciclos de ejecucin y tiempo de espera entre ciclos, expresado en minutos. Tambin dar la posibilidad de ejecutar detectar a continuacin, o bien recibir indicaciones de cmo hacerlo manualmente. La ejecucin de este comando deja inicializado el entorno de trabajo para la sesin de bash actual. Cualquier otra sesin de bash en que se quieran utilizar los comandos del trabajo, no representar un entorno vlido si no se ejecuta en ella el comando iniciar.
	
	\item [Cdigo fuente:]
\end{description}
{\footnotesize
\verbatiminput{../bin/iniciar}
}

\subsection{Comando Detectar}
\begin{description}
	\item [Tipo de comando:] Solicitado
	
	\item [Archivos de entrada:] Archivos ubicados en el directorio \verb|$DATADIR|
	
	\item [Archivos de salida:] Archivos de \verb|$DATADIR| que cumplen con el criterio de validez de nombres
	
	\item [Ejemplos de invocacin:]	El comando solo puede invocarse como
	\begin{verbatim}$ detectar\end{verbatim}
	de esta manera, lo que hace es verificar la validez de los archivos ubicados en el directorio \verb|$DATADIR|, para as colocar los correctos como archivos de entrada del intrprete en \verb|$DATADIR/ok|, y los incorrectos en \verb|$DATADIR/nok|.
	
	\item [Cdigo fuente:]
\end{description}
{\footnotesize
\verbatiminput{../bin/detectar}
}

\subsection{Comando Interprete}
\begin{description}
	\item [Tipo de comando:] Solicitado
	
	\item [Archivos de entrada:]
	\begin{enumerate}
		\item Archivos de Practico Recibidos en \verb|$DATADIR/ok|
		\item Tabla de Separadores \verb|$DATADIR/conf/T1.tab|
		\item Tabla de Campos \verb|$DATADIR/conf/T2.tab|
	\end{enumerate}
	
	\item [Archivos de salida:]
	\begin{enumerate}
		\item Archivo de Contratos de Préstamos Personales \verb|$DATADIR/new/CONTRAT.<pais>|
		\item Archivos (duplicados) Rechazados \verb|$DATADIR/nok/<nombre del archivo>|
		\item Archivos de Practico Procesados \verb|$DATADIR/old/<pais>-<sistema>-<año>-<mes>|
		\item Log \verb|$DATADIR/log/interprete.log|
	\end{enumerate}
	
	\item [Ejemplos de invocacin:]	El comando solo puede invocarse como
	\begin{verbatim}$ interprete\end{verbatim}
	de esta manera, lo que hace es procesar los archivos que se encuentran en el directorio \verb|$DATADIR/ok|, procesarlos y generar los archivos de contrato, con los cuales se llega a un formato estandar, denominados \verb|$DATADIR/new/CONTRAT.<pais>| en el directorio \verb|$DATADIR/new|. Los archivos ya procesados los deja en el directorio \verb|$DATADIR/old|, y los repetidos en \verb|$DATADIR/nok|.
	
	\item [Cdigo fuente:]
\end{description}
{\footnotesize
\verbatiminput{../bin/interprete}
}

\subsection{Comando X}
\begin{description}
	\item [Tipo de comando:] [Solicitado | Auxiliar]
	
	\item [Justificacin:] (de su uso si es auxiliar)
	
	\item [Archivos de entrada:]
	
	\item [Archivos de salida:]
	
	\item [Parmetros:] (si tiene)
	
	\item [Opciones:] (si tiene)
	
	\item [Ejemplos de invocacin:]
	
	\item [Cdigo fuente:]
\end{description}
% IMPORTANTE: Reemplazar tabulador por 4 espacios en los archivos fuente
{\footnotesize
%\verbatiminput{ruta relativa al archivo fuente}
}
%     
\subsection{Comando Reporte}
\begin{description}
	\item [Tipo de comando:] Solicitado
	
	\item [Archivos de entrada:]
		\begin{enumerate}
		\item Archivo de Contratos de Prstamos Personales ubicado en \verb|$grupo/datadir/new/CONTRAT.<pais>|
		\item Archivo Maestro Prstamos Personales Impagos ubicado en \verb|$grupo/datadir/mae/PPI.mae|
		\item Tabla de Paises y Sistemas ubicadas en \verb|$grupo/conf/p-s.tab|
		\end{enumerate}

	\item [Archivos de salida:] Los archivos de salida correspondientes a este comando, se generan si el usuario lo especifica. 
		\begin{enumerate}
		\item Log  \verb|$grupo/logdir/reporte.log|
		\item Listados \verb|$grupo/datadir/list/<nombre listado>.<id>|
		\item Archivo de Modificaciones de Contratos \verb|$grupo/datadir/new/MODIF.<pais>|
		\end{enumerate}

	\item [Ejemplos de invocacin:]	El comando solo puede invocarse como
	\begin{verbatim}$ reporte\end{verbatim}
	de esta manera, se carga un menu interactivo en donde el usuario puede:
		\begin{enumerate}
		\item Cargar parmetros de la consulta a realizar
		\item Activar o desactivar la grabacin de los listados de consulta
		\item Activar o desactivar la grabacin de modificaciones de contrato
		\item Realizar la consulta en base a los parmetros ingresados por teclado
		\end{enumerate}

	\item [Cdigo fuente:]
\end{description}
{\footnotesize
\verbatiminput{../bin/reporte.pl}
}

\section{Archivos}
\subsection{Archivo de configuracin general}
\begin{description}
	\item [Nombre:] \verb|$GRUPO/practico.conf|
	\item [Estructura:] Este archivo consta de lneas que pueden ser de alguno de los siguientes tipos:
	\begin{itemize}
		\item En blanco
		\item Comentario, comenzando la lnea con \verb|#| sin espacios previos
		\item Definicin de una variable en formato bash: \verb|VARIABLE=VALOR|. El valor puede contener espacios internos, si debe tener espacios al comienzo deben escapearse o encomillarse, aunque se recomienda especialmente no utilizar espacios en las rutas.
	\end{itemize}
	Este archivo debe tener como mnimo una serie de variables indispensables que el comando instalar generar. Por tal motivo se recomienda que en caso de modificacin no se eliminen variables previamente creadas, solo se las modifique y se agreguen variables nuevas.
\end{description}

\subsection{Archivo X}
\begin{description}
	\item [Nombre:] Su nombre
	\item [Estructura:] Descripcin de la estructura
	\item [Justificacin:] (de su uso si no es requerido por el enunciado)
\end{description}


\end{document}
